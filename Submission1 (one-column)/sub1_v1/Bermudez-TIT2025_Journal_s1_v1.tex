\documentclass[lettersize,onecolumn]{IEEEtran}
\usepackage[utf8]{inputenc} 
\usepackage[T1]{fontenc}
\usepackage{url}              % provides \url{...}
%\usepackage{ifthen}          % provides \ifthenelse
\usepackage{cite}             % improves presentation of citations

\usepackage[cmex10]{amsmath}	  % Use the [cmex10] option to ensure compliance
                              % with Xplore (see bare_conf.tex)
\usepackage{amsfonts}
\interdisplaylinepenalty=1000 % As explained in bare_conf.tex
\usepackage{mleftright}       % fix to wrong spacing of \left-,
                 			  % \middle- \right-commands
\usepackage{BermudezTIT2025}		  % The following package auto scales 
							  % (,{ and [ to be equivalent to \left(,
							  % \left{ and \left[ only in mathmode.
\mleftright




%\usepackage{algorithmic}
%\usepackage{algorithm}
%\usepackage{array}
%\usepackage[caption=false,font=normalsize,labelfont=sf,textfont=sf]{subfig}
%\usepackage{textcomp}
%\usepackage{stfloats}
%\usepackage{url}
%\usepackage{verbatim}
\usepackage{graphicx}
%\usepackage{orcidlink}

\hyphenation{op-tical net-works semi-conduc-tor -Xplore}
% updated with editorial comments 12/8/2023, 8/9/2021
\usepackage{orcidlink}
\usepackage{balance}


\begin{document}

\title{Title Journal}

%\author{Author 1{\orcidlink{0009-0009-2038-9985}},
%Author 2{\orcidlink{0000-0001-5597-1718}},
%Author 3{\orcidlink{0000-0002-1887-9215}}, and 
%Author 4{\orcidlink{0000-0002-2062-131X}}% <-this % stops a space
%\thanks{
%Author 1 is with the Department of Automatic Control \& Systems Engineering, The University of Sheffield, Sheffield S1 3JD, U.K.; and also with INRIA, Centre Inria d'Universit\'e C\^ote d'Azur, 06902 Sophia Antipolis, France (e-mail: jdaunastorres1@sheffield.ac.uk).
%
%Author 2 is with the Department of Automatic Control and Systems Engineering, The University of Sheffield, Sheffield S1 3JD, U.K.; and also with the Department of Electrical and Computer Engineering, Princeton University, Princeton, NJ 08544 USA (e-mail: esnaola@sheffield.ac.uk).
%
%Author 3 is with INRIA, Centre Inria d'Universit\'e C\^ote d'Azur, 06902 Sophia Antipolis, France; also with the Department of Electrical and Computer Engineering, Princeton University, Princeton, NJ 08544 USA; and also with the GAATI Mathematics Laboratory, University of French Polynesia, 98702 Faaa, French Polynesia (e-mail: samir.perlaza@inria.fr).
%
%Author 4 is with the Department of Electrical and Computer Engineering, Princeton University, Princeton, NJ 08544 USA (e-mail: poor@princeton.edu).
%
%This paper was presented in part at the  International Symposium on Information Theory (ISIT), Taipei, Taiwan, Jun., 2023 \cite{Perlaza-ISIT2023a}; and appears as an INRIA Technical Report in \cite{InriaRR9508}.
%}
%}


\maketitle

\begin{abstract}
Abstract of the Journal
\end{abstract}

\begin{IEEEkeywords}
Supervised Learning, Empirical Risk Minimization; Relative Entropy; Regularization; Gibbs Measure; Inductive Bias; Gibbs Algorithm; Sensitivity; and Generalization.
\end{IEEEkeywords}

\section{Introduction}
%-------------------------------------------------%
%
%           Empirical Risk Minimization
%
%-------------------------------------------------%
\section{Folkore Theorems of the Radom-Nikodym derivative}
\label{sec:FolkRND}
This section introduces relevant notational conventions alongside the Radon-Nikodym theorem.
%
In particular, some equalities presented in this paper are valid almost surely with respect to a given measure. For clarity, given a measure space $\left( \Omega, \mathscr{F}, P \right)$, the notation $\eqasP$ is introduced and shall be read as ``equal for all $x \in \Omega$ except on a negligible set with respect to $P$''; or equivalently as ``equal almost surely with respect to $P$''. Moreover, given two measures $P$ and $Q$ on the same measurable space, the notation $\abscontPQ$ stands for ``the measure $P$ is absolutely continuous with respect to $Q$''.
%
Using this notation, the Radon-Nikodym derivative is introduced by the following theorem.
%
\begin{theorem}[Radon-Nikodym theorem, {\cite[Theorem~2.2.1]{ash2000probability}}]\label{ThRNT}
Let~$P$ and~$Q$ be two measures on a given measurable space~$\left( \Omega, \mathscr{F}\right)$, such that $Q$ is $
\sigma$-finite and 
%$P$ is absolutely continuous with respect to~$Q$
$ \abscontPQ$. Then, there exists a nonnegative Borel measurable 
function~$g:~\Omega~\to~\mathbb{R}$ such that for all~$\mathcal{A}\in\mathscr{F}$, 
\begin{equation}\label{EqRNDa}
 P(\mathcal{A}) = \int_{\mathcal{A}} g(x) \mathrm{d}Q(x).
\end{equation}
Moreover, if another function $h$ satisfies for all~$\mathcal{A} \in \mathscr{F}$ that~$P(\mathcal{A}) = \int_{\mathcal{A}} h(x) 
\mathrm{d}Q(x)$, then $ g(x) \eqas{Q}$$ h(x)$. 
%almost surely with respect to $Q$.
\end{theorem}
The function $g$ in \eqref{EqRNDa} is often referred to as the Radom-Nikodym derivative of $P$ with respect to $Q$; and is also written as~$\frac{\mathrm{d}P}
{\mathrm{d}Q} $, such that~$g(x)~=~\frac{\mathrm{d}P}{\mathrm{d}Q} (x).$

The Radon-Nikodym theorem is the foundational tool from which many folklore theorems in information theory originate. Some of these folklore theorems are thoroughly studied in the following sections.
%This notation and theorem establish the foundational tools required for the subsequent results and analysis.
%\end{equation}
%In Theo 1 if not probability measure the Radon Nikodym derivative can be negative 
\section{Basic Folklore Theorems}
%Fundamental could replace basic here
This section focuses on basic folklore theorems, where ``basic'' denotes their well-established nature. One of the most common folklore theorems is often referred to as the ``change of measure'' theorem.
%In this section, proofs are presented for several well-known theorems, starting with the change of measure, which is among the most fundamental ones. 
%
\begin{theorem}[Change of Measure]\label{ThCOM}
Let $P$ and $Q$ be two measures on the measurable space $\left( \Omega, \mathscr{F}\right)$ with $\abscontPQ$; and $Q$ a $\sigma$-finite measure. Let $f: \Omega \to \reals$ be a Borel 
measurable function such that the integral $\int_{\Omega} f(x) \mathrm{d}P(x)$ exists. Then,  for all $
\mathcal{A} \in \mathscr{F}$,
\begin{equation}\label{EqTheoRNProperties1}
\displaystyle\int_{\mathcal{A}} f(x)\mathrm{d}P(x) = \displaystyle\int_{\mathcal{A}} f(x) \frac{\mathrm{d}P}{\mathrm{d}Q}(x)
 \mathrm{d}Q(x).
\end{equation} 
\end{theorem}
%
\begin{IEEEproof}
The first part of the proof is developed under the assumption that the function $f$ is simple. That is, for all $x \in \set{X}$, 
%
%To prove this theorem, the classical concept of simple function is used. Recall that a simple function $f$ is a function of the 
%form 
 $f(x) = \sum_{i=1}^{m} a_i \mathds{1}_{\mathcal{A}_i} (x)$, for finite $m \in \mathds{N}$, disjoint sets $\mathcal{A}_1$, $\mathcal{A}_2$, $\ldots$, $\mathcal{A}_m$ in $\mathscr{F}$  and reals~$a_1$,~$a_2$,~$\ldots$,~$a_m$. 
%From Theorem~\ref{ThRNT}, it holds that for all $\mathcal{A} \in \mathscr{F}$,
%\begin{equation}\label{TheoRNProperties1}
%P(\mathcal{A}) = \int_{\mathcal{A}}  \frac{\mathrm{d}P}{\mathrm{d}Q}(x)  \mathrm{d}Q(x).
%\end{equation}
For all $\mathcal{A} \in \mathscr{F}$, and for all $i \in \lbrace 1,2, \ldots, m \rbrace$, let  $\mathcal{B}_i = 
\mathcal{A} \cap \mathcal{A}_i$, hence, %for all $\set{A} \in \mathscr{F}$,
% not a partition anymore just disjoint : forming a partition of $\Omega$
% nonnegativity dropped
\begin{eqnarray}
%\nonumber
\label{EqMartinezPabon101}
\int_{\mathcal{A}} f(x)  \frac{\mathrm{d}P}{\mathrm{d}Q}(x) \mathrm{d} Q (x)
& = & \int_{\mathcal{A}} \frac{\mathrm{d}P}{\mathrm{d}Q} (x) \sum_{i=1}^{n} a_i \mathds{1}_{\mathcal{A}_i} (x) \mathrm{d} Q(x)\\
%
\label{EqMartinezPabon102}
%&  = &   \sum_{i=1}^{n} \int_{\mathcal{A}} a_i \frac{\mathrm{d}P}{\mathrm{d}Q}(x)  \mathds{1}_{\mathcal{A}_i }(x) \mathrm{d} Q(x) \\
%
%\label{EqMartinezPabon103}
%%&  = & \sum_{i=1}^{n} a_i \int_{\mathcal{A}} \frac{\mathrm{d}P}{\mathrm{d}Q}(x)  \mathds{1}_{\mathcal{A}_i (x)} \mathrm{d} Q(x)  \\
%
%\label{EqMartinezPabon104}
&  = & \sum_{i=1}^{n} a_i \int_{\mathcal{B}_i}  \frac{\mathrm{d}P}{\mathrm{d}Q}(x) \mathrm{d} Q(x) \\
\label{EqMartinezPabon105}
&  = & \sum_{i=1}^{n} a_i P (\mathcal{B}_i),
\end{eqnarray} 
where the equality in \eqref{EqMartinezPabon102} follows from the linearity of the integral \cite[Theorem~1.6.3]{ash2000probability}; 
and the equality in~\eqref{EqMartinezPabon105} follows from Theorem~\ref{ThRNT}. %\eqref{TheoRNProperties1}.
%
On the other hand, for all $\set{A} \in \mathscr{F}$
\begin{eqnarray}
\label{EqDiscoverMyReviewer101}
\displaystyle\int_{\mathcal{A}} f(x) \mathrm{d} P(x)  
&= & \displaystyle\int_{\mathcal{A}} \displaystyle\sum_{i=1}^{n} a_i \mathds{1}_{\mathcal{A}_i (x)} \mathrm{d} P(x) \\
\label{EqDiscoverMyReviewer102}
& =& \displaystyle\sum_{i=1}^{n} \int_{\mathcal{A}} a_i  \mathds{1}_{\mathcal{A}_i (x)}\mathrm{d} P(x)\\
%\label{EqDiscoverMyReviewer103}
%& =& \displaystyle\sum_{i=1}^{n} a_i \int_{\mathcal{A}} \mathds{1}_{\mathcal{A}_i (x)} \mathrm{d} Q(x)\\
%\label{EqDiscoverMyReviewer104}
&= & \sum_{i=1}^{n} a_i \int_{\mathcal{B}_i} \mathrm{d} P(x)%\\
\label{EqDiscoverMyReviewer105}
%& = &
= \sum_{i=1}^{n} a_i P (\mathcal{B}_i),
\end{eqnarray}
%
where the equality in \eqref{EqDiscoverMyReviewer102} follows from the linearity of the integral \cite[Theorem~1.6.3]
{ash2000probability}. Hence, from Theorem~\ref{ThRNT}, and equalities \eqref{EqMartinezPabon105} and \eqref{EqDiscoverMyReviewer105}, it follows that 
when~$f$ is a simple function, the equality in \eqref{EqTheoRNProperties1} holds.  This concludes the first part of the proof.

The second part of the proof proceeds by considering the following observations: $\left(a\right)$ simple functions form a dense subset of the space of Borel measurable functions  \cite[Theorem~1.5.5(b)]{ash2000probability}; and $\left(b\right)$ the integral is a continuous map from that space \cite[Theorem~1.6.2]{ash2000probability}. Hence, from $\left(a\right)$ and $\left(b\right)$, it follows that \eqref{EqTheoRNProperties1} also holds for any Borel measurable function~$f$.
%
This completes the proof.
\end{IEEEproof}
%
Another reputed  folklore theorem, which  is often referred to as the ``proportional measures’’ theorem, establishes the explicit forms of the Radon-Nikodym derivatives between two measures, in which one is proportional to the other.
%
\begin{theorem}[Proportional Measures]\label{TheoRNDOne}
Let $P$ and $Q$ be two $\sigma$-
finite measures on the measurable space $\left( \Omega, \mathscr{F}\right)$, such that for all $\set{A} \in \mathscr{F}$
\begin{eqnarray}\label{EqEqualMeasuresUpToConstant}
Q(\set{A}) = c P (\set{A}),
\end{eqnarray}
with $c > 0$. Then, for all $x \in \Omega$
\begin{eqnarray}
\label{EqThPropconst}
\frac{\mathrm{d}P}{\mathrm{d}Q} (x) &~\eqas{Q} \frac{1}{c} , \mbox{ and } \frac{\mathrm{d}Q}{\mathrm{d}P} (x) & \eqas{P}  c.
\end{eqnarray}
\end{theorem}
\begin{IEEEproof}
First, note that \eqref{EqEqualMeasuresUpToConstant} implies that the measures $P$ and $Q$ are mutually absolutely 
continuous. Hence, from Theorem~\ref{ThRNT}, it follows that for all $\mathcal{A} \in \mathscr{F}$, 
\begin{equation}
P\left( \mathcal{A} \right) = \int_{\mathcal{A}} \mathrm{d} P(x) = \int_{\mathcal{A}}   \frac{\mathrm{d}P}{\mathrm{d}Q}(x) \; 
\mathrm{d}Q(x).
\end{equation}
On the other hand, the equality \eqref{EqEqualMeasuresUpToConstant} also  implies 
\begin{equation}
P\left( \mathcal{A} \right) = \frac{1}{c} Q(\set{A}) =  \int_{\mathcal{A}} \frac{1}{c}  \mathrm{d} Q(x).
\end{equation}
Hence, it follows directly from Theorem~\ref{ThRNT} that the Radon-Nikodym derivative $ \frac{\mathrm{d} P}
{\mathrm{d} Q}$ is unique almost surely with respect to $Q$. Thus,
 $\frac{\mathrm{d}P}{\mathrm{d}Q}(x) \eqas{Q} $$\frac{1}{c}$.
Using similar arguments and the fact that $P$ and $Q$ are mutually absolutely continuous, it is verified that
 $\frac{\mathrm{d}Q}{\mathrm{d}P}(x) \eqasP$$ c$.
\end{IEEEproof}
 In the case in which $c=1$ in \eqref{EqThPropconst}, measures $P$ and $Q$ are identical, thus, $ \frac{\mathrm{d} P}{\mathrm{d} 
 Q}(x) \eqas{Q}$ $ \frac{\mathrm{d} Q}{\mathrm{d} P}(x) \eqasP $ $1.$
 
The following folklore theorem is often referred to as the ``chain rule''. 
 %
 \begin{theorem}[Chain Rule]\label{TheoChainsAndBlood}
Let $P$, $Q$, and $R$ be three measures on the measurable space $\left( \Omega, \mathscr{F}\right)$ such that 
$\abscontPQ$; 
$\abscont{Q}{R}$; 
and $Q$ and $R$ are $\sigma$-finite measures.  
Then, 
\begin{equation}\label{EqLimitedEditions1}
\frac{\mathrm{d}P}{\mathrm{d}R}(x) \eqas{R} \frac{\mathrm{d}P}{\mathrm{d}Q}(x)  \frac{\mathrm{d}Q}{\mathrm{d}R}(x).
\end{equation}
\end{theorem}
%
\begin{IEEEproof}
From the assumptions of the theorem, it follows that for all $\mathcal{A} \in \mathscr{F}$, 
\begin{eqnarray}
\label{EqChainRule2}
P (\mathcal{A}) 
& = & \int_{\mathcal{A}} \mathrm{d} P(x) 
= \int_{\mathcal{A}} \frac{\mathrm{d} P}{\mathrm{d} Q}(x) \mathrm{d} Q(x)\\
\label{EqChainRule3}
& = &  \int_{\mathcal{A}} \frac{\mathrm{d} P}{\mathrm{d} Q}(x)\frac{\mathrm{d} Q}{\mathrm{d} R}(x) \mathrm{d} R(x)\\
\label{EqChainRule4} 
& = & \int_{\mathcal{A}} \frac{\mathrm{d} P}{\mathrm{d} R}(x) \mathrm{d} R(x),
\end{eqnarray}
where 
the second equality in \eqref{EqChainRule2} follows from Theorem~\ref{ThRNT}; and
the equality in \eqref{EqChainRule3} follows from Theorem~\ref{ThCOM} . 
%
The equality in~\eqref{EqChainRule4} holds from Theorem~\ref{ThRNT} and by noticing that $\abscont{P}{R}$. 
%
Therefore, the equalities in \eqref{EqChainRule3} and \eqref{EqChainRule4} together with Theorem~\ref{ThRNT} imply~\eqref{EqLimitedEditions1}, which completes the proof.
\end{IEEEproof}
%
The following folklore theorem shows the connection between the Radon-Nikodym derivative and its multiplicative inverse.
%
\begin{theorem}[Multiplicative Inverse]\label{TheoInverseRND}
Let $P$ and $Q$ be two mutually absolutely continuous measures on the measurable space $\left( \Omega, \mathscr{F}\right)$; and assume that for all $x \in \Omega$,  $\frac{\mathrm{d}Q}{\mathrm{d}P}(x) > 0$. 
Then,  
\begin{eqnarray}\label{EqLimitedEditions}
\frac{\mathrm{d}P}{\mathrm{d}Q}(x) &  \eqas{Q} & \left( \frac{\mathrm{d}Q}{\mathrm{d}P}(x) \right)^{-1}.
\end{eqnarray}
\end{theorem}
\begin{IEEEproof}
From Theorem~\ref{TheoChainsAndBlood}, it follows that 
\begin{eqnarray}
\label{EqSomebodyCalls}
\frac{\mathrm{d} P}{\mathrm{d} Q} (x) \frac{\mathrm{d} Q}{\mathrm{d} P} (x)  \eqas{Q} \frac{\mathrm{d} Q}{\mathrm{d}Q}(x) \eqas{Q}  1,
\end{eqnarray}
%
where the last equality follows from Theorem~\ref{TheoRNDOne}, with~$c~=~1$. This completes the  proof.
\end{IEEEproof}
The subsequent folklore theorem establishes the linearity of the Radon-Nikodym derivative.
\begin{theorem}[Linearity]\label{TheoRNDadditivity}
Let~$P$ be a $\sigma$-finite measure on~$\left(\Omega, \mathscr{F}\right)$ and let also~$Q_1 , Q_2 , \ldots , Q_n$ be finite measures on~$\left(\Omega, \mathscr{F}\right)$ absolutely continuous with respect to~$P$. Let~$c_1 , c_2 , \ldots , c_n$ be positive reals; and let $S$ be a finite measure on $\left(\Omega, \mathscr{F}\right)$ such that for all~$\set{A} \in \mathscr{F}$, 
$S(\set{A}) = \sum_{t=1}^n c_t Q_t(\set{A})$.
Then,  
\begin{eqnarray}
\label{EqJanuary18at22h41inDubai}
\frac{\mathrm{d} S}{\mathrm{d}P}(x) & \eqasP & \sum_{t =1}^{n} c_t \frac{\mathrm{d}Q_t}{\mathrm{d}P} (x).
\end{eqnarray}
\end{theorem}
\begin{IEEEproof}
%
The proof starts by noticing that, from the assumptions of the theorem, it holds that $\abscont{S}{P}$. Hence, for all~$\set{A} \in \mathscr{F}$, it holds that
\begin{eqnarray} \label{EqProofTheoRNDadditivity2}
\label{EqChangeofmeasure105}
 & &  \int_{\mathcal{A}} \frac{\mathrm{d} S}{\mathrm{d}P}(x) \mathrm{d} P(x)  = \int_{\mathcal{A}}\mathrm{d} S(x)  =  \sum_{t=1}^n c_t Q_t(\mathcal{A}) \\ 
\label{EqChangeMeasure}
&= &\sum_{t=1}^n \int_{\mathcal{A}} c_t \mathrm{d} Q_t(x)   = \sum_{t=1}^n \int_{\mathcal{A}} c_t \frac{ \mathrm{d} Q_t}{\mathrm{d}P}(x) \mathrm{d} P(x) \\ 
\label{EqAddIntegral}
&= &\int_{\mathcal{A}} \sum_{t=1}^n c_t \frac{ \mathrm{d} Q_t}{\mathrm{d}P}(x) \mathrm{d} P(x),
\end{eqnarray}
where the first equality in~\eqref{EqChangeofmeasure105} and  the last equality in~\eqref{EqChangeMeasure} follow from Theorem~\ref{ThCOM}; and the equality 
\eqref{EqAddIntegral} follows from the additivity property of the integral \cite[Corollary~1.6.4]{ash2000probability}.
%
The proof ends by using Theorem~\ref{ThRNT}, which  implies the equality in~\eqref{EqJanuary18at22h41inDubai} from~\eqref{EqAddIntegral}.
\end{IEEEproof}
%
The following folklore theorem establishes the continuity of the Radon-Nikodym derivative.
%providing a critical result for analyzing the limits of absolutely continuous measures.
\begin{theorem}[Continuity]
Let~$P$ be a $\sigma$-finite measure on $\left(\Omega, \mathscr{F} \right)$, and let $Q_1, Q_2, \cdots$ be an infinite
sequence of $\sigma$-finite measures on $\left( \Omega, \mathscr{F} \right)$, converging to a measure $Q$. Suppose that for all $n \in \mathbb{N}$, $\abscont{Q_n}{P}$.
Then, $\abscont{Q}{P}$ and 
\begin{equation}\label{EqTheoRNDadditivity1}
\displaystyle\lim_{n \rightarrow \infty}  \frac{ \mathrm{d} Q_n}{\mathrm{d}P}(x) \eqasP  \frac{\mathrm{d}Q}{\mathrm{d}P}(x).
\end{equation}
\end{theorem}
\begin{IEEEproof}
From the assumptions of the theorem, for all  $\mathcal{A} \in \mathscr{F}$, it holds that
%
\begin{eqnarray}
\label{EqDixneufHeuresTreinte101}
Q (\mathcal{A}) 
& = & \lim_{n \rightarrow \infty}  Q _n (\mathcal{A}) \\
\label{EqDixneufHeuresTreinte102}
& =& \lim_{n \rightarrow \infty} \int_{\mathcal{A}}  \frac{ \mathrm{d} Q_n} {\mathrm{d}P}(x) \mathrm{d} P(x) \\
\label{EqDixneufHeuresTreinte103}
& = &  \int_{\mathcal{A}} \lim_{n \to \infty} \frac{ \mathrm{d} Q_n} {\mathrm{d}P}(x) \mathrm{d} P(x),
\end{eqnarray}
where 
the equality in \eqref{EqDixneufHeuresTreinte102} follows from Theorem~\ref{ThCOM}
; and 
the equality in \eqref{EqDixneufHeuresTreinte103} follows from \cite[Theorem~1.6.2]{ash2000probability}.
%
The equality in \eqref{EqDixneufHeuresTreinte103} implies that $\abscont{Q}{P}$. Hence, for all  $\mathcal{A} \in \mathscr{F}$, it holds that
\begin{equation}
\label{EqProofTheoRNDadditivity4}
Q (\mathcal{A}) = \int_{\mathcal{A}} \frac{ \mathrm{d} Q}  {\mathrm{d} P}(x) \mathrm{d} P(x).
\end{equation}
Therefore, the equalities in \eqref{EqDixneufHeuresTreinte103} and \eqref{EqProofTheoRNDadditivity4} jointly with Theorem~\ref{ThRNT} imply equation~\eqref{EqTheoRNDadditivity1}, which completes the proof.
\end{IEEEproof}
%
The ensuing folklore theorem establishes the relation between the Radon-Nikodym derivative of a product measure with respect to its component measures. 
 \begin{theorem}[Product of Measures]
For all $i \in \lbrace 1,2 \rbrace$, let $P_{i}$ and $Q_{i}$ be a finite and a $\sigma$-finite measure on $\left( \Omega_i, \mathscr{F}_i \right)$, respectively; with $\abscont{P_i}{Q_i}$.  Let also $P_1P_2$ and 
$Q_1 Q_2$ be the product measures on $\left( \Omega_1 \times \Omega_2, \mathscr{F}_1 \times \mathscr{F}_2 \right)$ formed by $P_1$ and $P_2$; and $Q_1$ and $Q_2$, 
respectively. Then, 
 \begin{eqnarray}
  \label{EqProductMeasure0}
\frac{\mathrm{d}P_{1} P_{2}}{\mathrm{d}Q_1Q_2}\left(x_1, x_2\right) &\eqas{Q_1Q_2}& \frac{\mathrm{d}P_{1}}{\mathrm{d}Q_1} \left(x_1\right)\frac{\mathrm{d}P_{2}}{\mathrm{d}Q_2}\left(x_2\right).
\end{eqnarray}
 \end{theorem}
\begin{IEEEproof}
From the assumptions of the theorem, for all $\set{A} \in \left(\Omega_1 \times\Omega_2 \right)$,
\begin{eqnarray}
 \label{EqProductMeasure1}
P_{1} P_{2}\left(\set{A}\right) &=& \int_{\set{A}} \mathrm{d}P_{1} P_{2}\left(x_1, x_2\right) \\
 \label{EqProductMeasure2}
&=& \int \int_{\set{A}_{x_2}} \mathrm{d}P_{1}\left(x_1\right) \mathrm{d}P_{2}\left(x_2\right)\\
 \label{EqProductMeasure3}
&=& \int \int_{\set{A}_{x_2}}\dfrac{\mathrm{d}P_{1}\left(x_1\right)}{\mathrm{d}Q_1} \mathrm{d}Q_1\left(x_1\right) \mathrm{d}P_{2}\left(x_2\right) \\
 \label{EqProductMeasure4}
&=& \int \int_{\set{A}_{x_2}} \dfrac{\mathrm{d}P_{1}\left(x_1\right)}{\mathrm{d}Q_1} \dfrac{\mathrm{d}P_{2}\left(x_2\right)}{\mathrm{d}Q_2} \mathrm{d}Q_1\left(x_1\right) \mathrm{d}Q_2\left(x_2\right) \\
 \label{EqProductMeasure5}
&=& \int_{\set{A}} \dfrac{\mathrm{d}P_{1}}{\mathrm{d}Q_1}\left(x_1\right) \dfrac{\mathrm{d}P_{2}}{\mathrm{d}Q_2}\left(x_2\right) \mathrm{d}Q_1Q_2\left(x_1, x_2\right), 
\end{eqnarray}
where~$\set{A}_{x_2}$ is the section of the set~$\set{A}$ determined by~$x_2$, namely, $\set{A}_{x_2} \triangleq \left\lbrace x_1 \in \Omega_1: (x_1,x_2) \in \set{A} \right\rbrace$;  the equality in \eqref{EqProductMeasure1} arises from the definition of $P_1P_2$  as the product of $P_1 $ and  $P_2$; 
the equality in \eqref{EqProductMeasure3} is a direct consequence of Theorem~\ref{ThCOM};  
the equality in \eqref{EqProductMeasure4} follows from Theorem~\ref{ThRNT}; and finally, the equality in \eqref{EqProductMeasure5} is due to the construction of  $Q_1Q_2$ as the product measure of  $Q_1$ and $Q_2$.

The proof follows by observing that from the equality in \eqref{EqProductMeasure5}, it holds that $\abscont{P_1P_2}{Q_1Q_2}$. Thus, for all~$\set{A}~\in~\mathscr{F}_1~\times~\mathscr{F}_2$, 
\begin{eqnarray}
P_{1} P_{2}\left(\set{A}\right) 
%&=& \int_{\set{A}} \mathrm{d}P_{1} P_{2}\left(x_1, x_2\right) \\
\label{EqCartProduct1}
&=& \int_{\set{A}} \dfrac{\mathrm{d}P_{1} P_{2}}{\mathrm{d}Q_1Q_2}\left(x_1, x_2\right) \mathrm{d}Q_1Q_2\left(x_1, x_2\right).
\end{eqnarray}
%
The equalities in \eqref{EqProductMeasure5} and \eqref{EqCartProduct1}, together with Theorem~\ref{ThRNT}, imply the equality in~\eqref{EqProductMeasure0}, which completes the proof.
%almost surely with respect to $Q_1Q_2$, which completes the proof.
%
%From Theorem~\ref{ThRNT}, it follows that the equalities in \eqref{EqProductMeasure5} and \eqref{EqCartProduct1} hold almost surely with respect to $Q_1Q_2$, thereby completing the proof.
\end{IEEEproof}

\section{Advanced folklore theorems}
This section requires some additional notation. In particular, denote by $\triangle\left(\mathcal{X},\mathscr{F}_\mathcal{X}\right)$, or simply~$\triangle\left(\mathcal{X}\right)$, the set of all probability measures on the measurable space $\left(\mathcal{X},\mathscr{F}_{\set{X}}\right)$, where $\mathscr{F}_{\set{X}}$ is a $\sigma$-algebra on $\set{X}$.
%
Using this notation, conditional probability measures can be defined as follows. 
\begin{definition}[Conditional Probability]
A family $P_{Y|X}~\triangleq~(P_{Y|X=x})_{x\in\set{X}}$ of elements of $\triangle( \set{Y}, \mathscr{F}_{\set{Y}})$ indexed by $\set{X}$ is said to be a conditional probability measure if, for all sets $\mathcal{A} \in \mathscr{F}_{\set{Y}}$, the map
\begin{eqnarray}
\set{X}& \to & [0, 1] \\
x & \mapsto & P_{Y|X=x}(\mathcal{A})
\end{eqnarray}
is Borel measurable. The set of such conditional probability measures is denoted by $\triangle(\set{Y}|\set{X})$.
\end{definition}

A conditional probability $P_{Y|X} \in \triangle(\mathcal{Y} | \mathcal{X})$ and a probability measure $P_X \in \triangle(\mathcal{X})$ determine two unique probability measures in $\triangle(\mathcal{X} \times \mathcal{Y})$ and $\triangle(\mathcal{Y} \times \mathcal{X})$, respectively. These probability measures are denoted by $P_{XY}$ and $P_{YX}$, respectively, and for all sets $\mathcal{A} \in \mathscr{F}_{\set{X}} \times \mathscr{F}_{\set{Y}}$, it follows that
%
\begin{eqnarray}
\label{EqMay20at14h23in2024}
P_{X Y }\left( \set{A} \right) & = & \int P_{Y |X =x} \left( \set{A}_{x}\right) \mathrm{d} P_{X} \left(x\right),
\end{eqnarray}
where~$\set{A}_{x}$ is the section of the set~$\set{A}$ determined by~$x$, namely, 
\begin{eqnarray}
\label{EqNovember15at16h57in2024InTheBusToNice}
\set{A}_{x} \triangleq \left\lbrace y \in \set{Y}: (x,y) \in \set{A} \right\rbrace.
\end{eqnarray}
%
Alternatively,  for all sets~$\set{B} \in \mathscr{F}_{\set{Y}} \times \mathscr{F}_{\set{X}}$, it follows that
\begin{eqnarray}
\label{EqOctober30at7h48in2024SophiaAntipolis}
P_{ YX } \left( \set{B} \right) & = & \int P_{Y |X =x} \left( \set{B}_{x}\right) \mathrm{d} P_{X} \left(x\right),
\end{eqnarray}
where~$\set{B}_{x}$ is the section of the set~$\set{B}$ determined by~$x$. 
%
For all sets~$\set{A} \in  \mathscr{F}_{\set{X}} \times \mathscr{F}_{\set{Y}}$, let the set~$\hat{\set{A}} \in  \mathscr{F}_{\set{Y}} \times \mathscr{F}_{\set{X}}$ be such that 
\begin{eqnarray}
\label{EqNovember16at18h49in2024Nice}
\hat{\set{A}}  & = & \left\lbrace (y, x) \in\set{Y} \times \set{X}:  (x , y) \in \set{A} \right\rbrace.
\end{eqnarray}
Then, from~\eqref{EqMay20at14h23in2024} and~\eqref{EqOctober30at7h48in2024SophiaAntipolis}, it holds that
\begin{eqnarray}\label{EqJanuary10at16h43in2025}
P_{XY }\left( \set{A} \right) & = & P_{YX}\left( \hat{\set{A}} \right).
\end{eqnarray}
%
Using this notation, the notion of marginal probability measures can be introduced as follows. 
%capturing the individual behavior of a random variable by "marginalizing out" the other variables.
\begin{definition}[Marginals]
Given two joint probability measures $P_{XY} \in \triangle\left( \set{X} \times \set{Y} \right)$ and $P_{YX} \in \triangle\left( \set{Y} 
\times \set{X} \right)$, satisfying \eqref{EqJanuary10at16h43in2025}, the marginal probability measures in $\triangle\left( \set{X} 
\right)$ and $\triangle\left( \set{Y} \right)$, denoted by $P_X$ and $P_Y$, respectively satisfy for all  sets $\set{A} \in  \mathscr{F}_{\set{X}}$ and for all sets $\set{B} \in \mathscr{F}_{\set{Y}}$,
\begin{eqnarray}
\label{eqmargpx}
P_{X}\left( \set{A} \right) & \triangleq  P_{XY}\left( \set{A} \times \set{Y} \right) = & P_{YX}\left( \set{Y} \times \set{A} 
\right) ; \mbox{ and } \\
\label{eqmargpy}
P_{Y}\left( \set{B} \right) & \triangleq  P_{XY}\left( \set{X} \times \set{B} \right) = & P_{YX}\left( \set{B} \times \set{X} 
\right).
\end{eqnarray}
\end{definition}
From the total probability theorem \cite[Theorem~4.5.2]{ash2000probability}, it follows that for all~$\set{A} \in \mathscr{F}
_{\set{Y}}$,  
\begin{eqnarray}\label{eqmarginal1}
%\nonumber
 P_Y(\set{A})  = \int \int_{\set{A}} \mathrm{d}P_{Y | X = x}(y) \mathrm{d}P_X (x);
\end{eqnarray}
%$P_Y(A)$ is the marginal of $P_{Y|X} P_X$. For all~$\set{B} \in \mathscr{F}_{\set{X}}$, the marginal of $P_{X|Y} P_Y$ is 
and for all~$\set{B} \in \mathscr{F}_{\set{X}}$, 
\begin{eqnarray}\label{eqmarginal2}
P_X(\set{B}) =    \int \int_{\set{B}} \mathrm{d}P_{X | Y = y}(x) \mathrm{d}P_Y(y).
\end{eqnarray}
%
The joint probability measures~$P_{XY}$ and~$P_{YX}$ can be described via the conditional probability measure~$P_{Y | X}$ 
and the probability measure~$P_{X}$ as in~\eqref{EqMay20at14h23in2024} and 
in~\eqref{EqOctober30at7h48in2024SophiaAntipolis}; or via the conditional probability measure~$P_{X | Y} \in \triangle\left( \set{X} | \set{Y} \right)$ and the marginal probability 
measure~$P_{Y} \in \triangle\left( \set{Y} \right)$.
%
More specifically, for all sets~$\set{A} \in  \mathscr{F}_{\set{X}} \times \mathscr{F}_{\set{Y}}$, it follows that
\begin{eqnarray}
\label{EqJun3at14h31in2024}
P_{XY }\left( \set{A} \right) & = & \int P_{X | Y = y} \left( \set{A}_{y}\right) \mathrm{d} P_{Y} \left( y\right),
\end{eqnarray}
where~$\set{A}_{y}$ is the section of the set~$\set{A}$ determined by~$y$, namely, 
\begin{eqnarray}
\label{EqNovember15at16h39in2024InTheBusToNice}
\set{A}_{y} \triangleq \left\lbrace x \in \set{X} : (x,y) \in \set{A} \right\rbrace.
\end{eqnarray}
%A similar approach can be used for $P_{YX}$.
Alternatively, for all sets~$\set{B} \in \mathscr{F}_{\set{Y}} \times \mathscr{F}_{\set{X}}$, it follows that
\begin{eqnarray}
\label{EqNovember10at20h22in2024Nice}
P_{ YX } \left( \set{B} \right) & = & \int P_{X | Y = y} \left( \set{B}_{y}\right) \mathrm{d} P_{Y} \left( y\right),
\end{eqnarray}
where~$\set{B}_{y}$ is the section set of~$\set{B}$ determined by $y$.

Within this context, the following folklore theorem highlights a property of conditional measures, which is reminiscent of the unit measure axiom in probability theory. 
\begin{theorem}[Unit Measure]\label{ThUnitMeasure}
%CHANGe name 
Consider the conditional probability measures~$P_{Y | X} \in \triangle\left( \set{Y} | \set{X} \right)$ and~$P_{X | Y}\in \triangle\left( \set{X} | \set{Y} \right)$; the probability measures~$P_{Y} \in \triangle\left( \set{Y} \right)$ 
and~$P_{X} \in \triangle\left(  \set{X} \right)$ that satisfy~\eqref{eqmarginal1} and~\eqref{eqmarginal2}. Assume that for all~$x \in \set{X}$, the probability measure~$\abscont{P_{Y | X = x}}{P_{Y}}$. Then,
\begin{eqnarray}
\label{EqThCondPorp}
\int  \frac{\mathrm{d} P_{Y|X = x}}{\mathrm{d}P_{Y}} (y) \mathrm{d}P_{X}(x) &\eqas{P_Y}& 1.
 \end{eqnarray}
 \end{theorem}
% 
\begin{IEEEproof}
For all $\set{A} \in \mathscr{F}_{\set{Y}}$, from \eqref{eqmarginal1}, it holds that
\begin{eqnarray}
\label{EqDefOfMarg}
 P_{Y}(\set{A}) & =& \int \int_{\set{A}} \mathrm{d}P_{Y | X = x}(y) \mathrm{d}P_X (x)\\
 \label{EqIDK}
  & = & \int \int_{\set{A}} \frac{\mathrm{d} P_{Y|X = x}}{\mathrm{d}P_{Y}}(y) \mathrm{d}P_{Y}(y) \mathrm{d}P_X (x)\\
  \label{EqResult}
  & =& \int_{\set{A}} \int  \frac{\mathrm{d} P_{Y|X = x}}{\mathrm{d}P_{Y}}(y) \mathrm{d}P_X (x) \mathrm{d}P_{Y}(y),
 \end{eqnarray}
 %
where the equality in  \eqref{EqIDK} follows from a change of measure (Theorem~\ref{ThCOM}). Moreover, \eqref{EqResult} is 
obtained using Fubini's theorem \cite[Theorem~2.6.6]{ash2000probability}.
The proof proceeds by noticing that $P_{Y}(\set{A})  =\int _{\set{A}} \mathrm{d}P_{Y}(y)$, and thus from Theorem~\ref{ThRNT} and the equality in \eqref{EqResult}, the statement in \eqref{EqThCondPorp} holds.
\end{IEEEproof}

 The following folklore theorem is reminiscent of the Bayes rule.

\begin{theorem}[Bayes-like rule]\label{TheoBYR}
Consider the conditional probability measures~$P_{Y | X}$ and~$P_{X | Y}$; the probability measures~$P_{Y}$ 
and~$P_{X}$ that satisfy~\eqref{eqmarginal1} and~\eqref{eqmarginal2}; and the joint probability measures~$P_{YX}$ and~$P_{XY}$ in \eqref{EqOctober30at7h48in2024SophiaAntipolis} and \eqref{EqJun3at14h31in2024} respectively. 
%
Let also $P_{X}P_{Y} \in \triangle\left( \set{X} \times \set{Y} , \mathscr{F}_{\set{X}} \times \mathscr{F}_{\set{Y}}\right)$ and $P_{Y}P_{X} \in \triangle\left( \set{Y} \times \set{X} , \mathscr{F}_{\set{Y}} \times \mathscr{F}_{\set{X}}\right)$ be the measures formed by the product of the marginals $P_{X}$ and $P_{Y}$. 
%
 Assume that:
\begin{itemize}
\item[$\left(a\right)$] For all~$x \in \set{X}$,~$\abscont{P_{Y | X = x}}{P_{Y}}$; 
and 
\item[$\left(b\right)$] For all~$y \in \set{Y}$,~$\abscont{P_{X | Y = y}}{P_{X}}$.
\end{itemize}
Then,
% for all~$\left( x , y \right) \in \left( \set{X} \times \set{Y}\right)$,  
\begin{eqnarray}
\label{EqOuter1Inner1}
\frac{\mathrm{d} P_{XY}}{\mathrm{d} P_{X}P_{Y}} \left( x,y \right) &\eqas{P_XP_Y}& \frac{\mathrm{d} P_{X | Y = y}}{\mathrm{d} P_{X}} \left( x \right)\\
 \label{EqInner1Inner2}
 & \eqas{P_XP_Y} & \frac{\mathrm{d} P_{Y | X = x}}{\mathrm{d} P_{Y}}\left( y \right)\\
 \label{EqInner2Outer2}
 & \eqas{P_XP_Y} & \frac{\mathrm{d} P_{YX}}{\mathrm{d} P_{Y}P_{X}}\left( y,x\right).
\end{eqnarray}
\end{theorem}
%
\begin{IEEEproof}
%The proof begins by considering the joint probability measure $P_{XY}$ 
Note that assumptions $\left(a\right)$ 
and $\left(b\right)$ are sufficient for the Radon-Nikodym derivatives of $P_{XY}$ with respect to $P_XP_Y$
and~$P_{YX}$ with respect to $P_YP_X$ to exist. Hence, it follows that for all sets~$\set{A}~\in~ \mathscr{F}_{\set{X}} \times \mathscr{F}_{\set{Y}}$,
\begin{eqnarray}
\label{EqJanuary19at21h57inDubai}
P_{XY }\left( \set{A} \right) & = & \int_{\set{A}} \frac{\mathrm{d}P_{XY}} {\mathrm{d}P_{X}P_{Y}} (x,y) \mathrm{d}P_{X}P_{Y} (x,y),   
\end{eqnarray}
which follows from Theorem~\ref{ThCOM}.
Note also that from~\eqref{EqJun3at14h31in2024}, it follows that
%The proof proceeds by noticing that from~\eqref{EqJun3at14h31in2024}, it holds that
\begin{eqnarray}
\label{EqOctober31at13h12in2024Nicea}
P_{X Y }\left( \set{A} \right) & = & \int \int_{ \set{A}_{y} } \mathrm{d} P_{X | Y = y} \left( x \right) \mathrm{d} P_{Y} \left( y\right)\\
\label{EqOctober31at13h12in2024Niceb}
&=& \int \int_{ \set{A}_{y} } \frac{\mathrm{d} P_{X | Y = y} }{\mathrm{d}P_{X}} \left( x \right) \mathrm{d}P_{X}\left( x \right) 
\mathrm{d} P_{Y} \left( y\right)  \\
\label{EqNovember15at16h25in2024InTheBusToNiceA}
& = & \int \int \mathds{1}_{\set{A}_y}(x) \frac{\mathrm{d} P_{X | Y = y} }{\mathrm{d}P_{X}} \left( x \right)  \mathrm{d}P_{X}\left( x \right) 
\mathrm{d} P_{Y} \left( y\right)  \\
\label{EqNovember15at16h25in2024InTheBusToNiceB}
& = & \int \mathds{1}_{\set{A}}(x,y) \frac{\mathrm{d} P_{X | Y = y} }{\mathrm{d}P_{X}} \left( x \right)  \mathrm{d} P_{X} P_{Y} \left(x, 
y\right)  \\
\label{EqOctober31at13h12in2024Nicec}
& = & \int_{\set{A}} \frac{\mathrm{d} P_{X | Y = y} }{\mathrm{d}P_{X}} \left( x \right) \mathrm{d} P_{X} P_{Y} \left(x,  y\right), 
\end{eqnarray}
%
where, the set~$\set{A}_{y}$ is defined in~\eqref{EqNovember15at16h39in2024InTheBusToNice}.  Moreover, the equality 
in~\eqref{EqOctober31at13h12in2024Niceb} follows from Assumption~$\left(b\right)$ and Theorem \ref{ThRNT}. 
%
Similarly, from~\eqref{EqMay20at14h23in2024}, it follows that 
%
\begin{eqnarray}
\label{EqNovember15at16h36in2024InTheBusToNiceA}
P_{X Y }\left( \set{A} \right)  & = & \int \int_{ \set{A}_{x} } \mathrm{d} P_{Y | X = x} \left( y \right) \mathrm{d} 
P_{X} \left( x\right)\\
\label{EqNovember15at16h36in2024InTheBusToNiceB}
& = & \int \int_{ \set{A}_{x} } \frac{\mathrm{d} P_{Y | X = x} }{\mathrm{d}P_{Y}} \left( y \right)\mathrm{d} P_{Y} \left( y\right) \mathrm{d}P_{X}\left( x \right)   \\
\label{EqNovember15at16h36in2024InTheBusToNiceC}
& = & \int \int \mathds{1}_{\set{A}_x}(y) \frac{\mathrm{d} P_{Y | X = x} }{\mathrm{d}P_{Y}} \left( y \right) \mathrm{d} P_{Y} \left( y\right) 
\mathrm{d}P_{X}\left( x \right)   \\
\label{EqNovember15at16h36in2024InTheBusToNiceC1}
& = & \int \int \mathds{1}_{\set{A}_y}(x) \frac{\mathrm{d} P_{Y | X = x} }{\mathrm{d}P_{Y}} \left( y \right)  \mathrm{d}P_{X}\left( x \right) 
\mathrm{d} P_{Y} \left( y\right)    \\
\label{EqNovember15at16h36in2024InTheBusToNiceD}
& = & \int \mathds{1}_{\set{A}}(x,y) \frac{\mathrm{d} P_{Y | X = x} }{\mathrm{d}P_{Y}} \left( y \right) \mathrm{d} P_{X}P_{Y}  \left(  x , 
y\right)   \\\label{EqNovember15at16h36in2024InTheBusToNiceE}
& = & \int_{\set{A}} \frac{\mathrm{d} P_{Y | X = x} }{\mathrm{d}P_{Y}} \left( y \right)  \mathrm{d} P_{X}P_{Y}  \left(  x , y\right),  
\end{eqnarray}
where, the set~$\set{A}_{x}$ is defined in~\eqref{EqNovember15at16h57in2024InTheBusToNice}.  Moreover, the equality 
in~\eqref{EqNovember15at16h36in2024InTheBusToNiceB} follows from Assumption~$\left(a\right)$ and Theorem \ref{ThRNT}; and
the equality in~\eqref{EqNovember15at16h36in2024InTheBusToNiceC1} follows by exchanging the order of integration \cite[Theorem~2.6.6]{ash2000probability}.
Finally, from~\eqref{EqJanuary10at16h43in2025}, it follows that
\begin{eqnarray}
\nonumber
P_{XY}\left( \set{A} \right) 
&=& \int_ {\hat{\set{A}} } \mathrm{d} P_{YX}\left(y,x\right)\\
\label{EqAntibesTT1}
&=& \int_{\hat{\set{A}}}\frac{ \mathrm{d}P_{YX}} { \mathrm{d} P_YP_X}\left(y,x\right) \mathrm{d}P_YP_X\left(y,x\right)\\ 
&=& \int \mathds{1}_{\hat{\set{A}}}(y,x) \frac{ \mathrm{d}P_{YX}} { \mathrm{d} P_YP_X}\left(y,x\right) \mathrm{d}P_YP_X\left(y,x\right)  \\
\label{EqAntibesTT2}
&=&  \int \int \mathds{1}_{\hat{\set{A}}_x}(y)\frac{ \mathrm{d}P_{YX}} { \mathrm{d} P_YP_X}\left(y,x\right) \mathrm{d}P_Y(y) \mathrm{d}P_X(x) \\
\label{EqAntibesTT3}
&=& \int \int \mathds{1}_{\hat{\set{A}}_y}(x)\frac{ \mathrm{d}P_{YX}} { \mathrm{d} P_YP_X}\left(y,x\right) \mathrm{d}P_X(x) \mathrm{d}P_Y(y)  \\
&=& \int \mathds{1}_{\hat{\set{A}}}(x,y)\frac{ \mathrm{d}P_{YX}} { \mathrm{d} P_YP_X}\left(y,x\right) \mathrm{d}P_XP_Y(x,y)\\ 
\label{EqAntibesin2025Jan}
&=& \int_{\set{A}} \frac{ \mathrm{d}P_{YX}} { \mathrm{d} P_YP_X}\left(y,x\right) \mathrm{d}P_XP_Y(x,y), 
\end{eqnarray}
where the set $\hat{\set{A}}$ is defined in \eqref{EqNovember16at18h49in2024Nice}. By performing a change of measure using Theorem \ref{ThCOM} and assumption $\left(a\right)$, the equality in \eqref{EqAntibesTT1} is obtained; and the equality in \eqref{EqAntibesTT3} follows by exchanging the order of integration \cite[Theorem~2.6.6]{ash2000probability}. 

The proof is completed from Theorem~\ref{ThRNT} and by combining equations~\eqref{EqJanuary19at21h57inDubai}, ~\eqref{EqOctober31at13h12in2024Nicec}, ~\eqref{EqNovember15at16h36in2024InTheBusToNiceE} and~\eqref{EqAntibesin2025Jan}, which establish \eqref{EqOuter1Inner1}, \eqref{EqInner1Inner2} and~\eqref{EqInner2Outer2}.
\end{IEEEproof}

\begin{example}
%This exemples shows that in (a) the asumption on all x is necessary. Indeed, lets prove that there exist some x in omega such that PY given X is 0 and Py is not 0 in this case the absolute continuity assumption is not true therefore the random Nikiodym derivation of PY|x py does not exists. 

From definition \ref{DefAbsCountinuity}, we want to show that there exist an $x \in \set{X}$ such that $P_{Y|X=x}(A) > 0$ and $P_X(\{x \in \set{X} : P_{Y|X=x} (A) > 0\} )= 0$

For all $x \in \mathbb{R}$, for $\set{A} \in \mathscr{F}_Y$, for $\set{B} \in \mathscr{F}_Y$ and $\mu$ the Lebesgue measure, $P_{Y|X=x}(A)$ and
$P_X (B)$ are defined as follows 
\begin{IEEEeqnarray}{rCl}
P_{Y|X=x}(A) &=& \mu(A \cap (-x, x))\\
P_X (B) &=& \mu(B)
\end{IEEEeqnarray}
%and 
%\begin{IEEEeqnarray}{rCl}
%P_X (B) &=& \mu(B)
%\end{IEEEeqnarray}

Given $A \in \set{B}(\mathbb{R})$ such that $P_Y(A) = 0$, then 
\begin{IEEEeqnarray}{rCl}
P_Y(A) &= &\int P_{Y|X=x}(A) \mathrm{d}P_X(x)\\
\label{EqMadrid9am59}
& = & \int \mu \left(A \cap \left(-x, x\right)\right)  \ind{x \in  \mathbb{N}} \mathrm{d}P_X(x)\\ 
\label{EqMadrid10am00}
&=& \int_\mathbb{N} \mu \left(A \cap \left(-x, x\right)\right) \mathrm{d}P_X(x)\\
\label{EqMadrid10am01}
& = & \sum_{t=0}^\infty \mu \left(A \cap \left(-t, t\right)\right) \mu(\{t\})\\
\label{EqMadrid10am02}
& = & 0 \label{EqMadrid10am03}
\end{IEEEeqnarray}
The equality in \eqref{EqMadrid9am59} follows from \eqref{eqmarginal1}; the equality \eqref{EqMadrid10am03} follows from the definition of the Lebesgue measure of a singleton. Thus, this exemple constructs $P_{Y|X}$ and $P_X$ such that for some $\set{A} \in \set{B}(\mathbb{R})$ $P_Y(A) = 0$ but $P_{Y|X=x}(A) > 0$  for some $x \in \set{X}$ , therefore the Radom-Nikodym derivative of $P_Y|X=x$ with respect to $P_Y$ does not exist. This proves that assumption (a) in \ref{TheoBYR} cannot be weakened and is needed for all $x \in \set{X}$. A example for assumption (b) in \ref{TheoBYR} can be constructed similarly. 
\end{example}


\begin{theorem}[Inverse Bayes-like Rule]\label{TheoBYR2}
Consider the conditional probability measures~$P_{Y | X}$ and~$P_{X | Y}$;  and the probability measures~$P_{Y}$ 
and~$P_{X}$ that satisfy~\eqref{eqmarginal1} and~\eqref{eqmarginal2}; and the joint probability measures~$P_{YX}$ and~$P_{XY}$ in~\eqref{EqOctober30at7h48in2024SophiaAntipolis} and~\eqref{EqJun3at14h31in2024}  respectively.
 Assume that:
\begin{itemize}
\item[$\left(a\right)$] For all~$x \in \set{X}$,~$\abscont{P_{Y}}{P_{Y | X = x}}$; 
and 
\item[$\left(b\right)$] For all~$y \in \set{Y}$,~$\abscont{P_{X}}{P_{X | Y = y}}$. 
\end{itemize} 
Then, 
\begin{eqnarray}
 \label{EqInverseOuter1Inner1}
\frac{\mathrm{d} P_{X}P_{Y}}{\mathrm{d} P_{XY}} \left( x,y \right)& \eqas{P_{XY}}& \frac{\mathrm{d} P_{X}}{\mathrm{d} P_{X | Y = y}} \left( x \right)\\
 \label{EqInverseInner1Inner2}
 & \eqas{P_{XY}} & \frac{\mathrm{d} P_{Y}}{\mathrm{d} P_{Y | X = x}}\left( y \right)\\
  \label{EqInverseInner2Outer2}
 & \eqas{P_{YX}}&  \frac{\mathrm{d} P_{Y}P_{X}}{\mathrm{d} P_{YX}}\left( y,x\right).
\end{eqnarray}
\end{theorem}
\begin{IEEEproof}
The proof follows along the same lines as the proof of Theorem~\ref{TheoBYR}.
The proof follows by observing that for all measurable sets~$\set{A} \in \set{X} \times \set{Y}$,  the product measure~$P_{X}P_{Y} \in \triangle\left( \set{X} \times \set{Y} \right)$ satisfies
\begin{eqnarray}
\nonumber
& & P_{X}P_{Y}  \left(\set{A} \right)  \\
\label{EqNovember16at18h29in2024NiceA}
& = &\int_{\set{A}} \mathrm{d}P_{X}P_{Y}  \left( x, y\right) \\
\label{EqNovember16at18h29in2024NiceB}
& = &\int \int_{\set{A}_{y}} \mathrm{d}P_{X}\left( x\right)  \mathrm{d}P_{Y}  \left(y\right) \\
\label{EqNovember16at18h29in2024NiceC}
& = &\int \int_{\set{A}_{y}}  \frac{\mathrm{d}P_{X}}{\mathrm{d}P_{X | Y = y}} \left( x \right) \mathrm{d}P_{X | Y = y} \left( x \right)  \mathrm{d}P_{Y}  \left(y\right) \spnum \\
\label{EqNovember16at18h29in2024NiceD}
& = &\int \int \frac{\mathrm{d}P_{X}}{\mathrm{d}P_{X | Y = y}} \left( x \right) \ind{x \in \set{A}_{y}}  \mathrm{d}P_{X | Y = y} \left( x \right)  \mathrm{d}P_{Y}  \left(y\right) \spnum \\
\label{EqNovember16at18h29in2024NiceE}
& = &\int \frac{\mathrm{d}P_{X}}{\mathrm{d}P_{X | Y = y}} \left( x \right) \ind{x \in \set{A}_{y}}  \mathrm{d}P_{XY} \left( x, y\right) \spnum \\
\label{EqNovember16at18h29in2024NiceF}
& = &\int_{\set{A}} \frac{\mathrm{d}P_{X}}{\mathrm{d}P_{X | Y = y}} \left( x \right)    \mathrm{d}P_{XY} \left( x, y\right), \spnum 
\end{eqnarray}
where the set~$\set{A}_{y}$ is defined in~\eqref{EqNovember15at16h39in2024InTheBusToNice}; the equality in~\eqref{EqNovember16at18h29in2024NiceC} follows from Assumption~$(b)$ and \cite[Theorem~$2.2.3$]{lehmann2005testing}; and the measure~$P_{XY}$ is defined in~\eqref{EqJun3at14h31in2024}.

The proof proceeds by noticing that for all measurable sets~$\set{A} \in \set{X} \times \set{Y}$,  the product measure~$P_{X}P_{Y} \in \triangle\left( \set{X} \times \set{Y} \right)$ also satisfies
\begin{eqnarray}
\nonumber
& & P_{X}P_{Y}  \left(\set{A} \right)  \\
\label{EqNovember16at18h36in2024NiceA}
& = &\int_{\set{A}} \mathrm{d}P_{X}P_{Y}  \left( x, y\right) \\
\label{EqNovember16at18h36in2024NiceB}
& = &\int \int_{\set{A}_{y}} \mathrm{d}P_{X}\left( x\right)  \mathrm{d}P_{Y}  \left(y\right) \\
\label{EqNovember16at18h36in2024NiceC}
& = &\int \int_{\set{A}_{x}}  \mathrm{d}P_{Y}  \left(y\right) \mathrm{d}P_{X}\left( x\right)  \\
\label{EqNovember16at18h36in2024NiceC1}
& = &\int \int_{\set{A}_{x}}  \frac{\mathrm{d}P_{Y}}{\mathrm{d}P_{Y | X = x}} \left( y \right) \mathrm{d}P_{Y | X = x} \left( y \right)  \mathrm{d}P_{X}  \left(x\right) \spnum \\
\label{EqNovember16at18h36in2024NiceD}
& = &\int \int  \frac{\mathrm{d}P_{Y}}{\mathrm{d}P_{Y | X = x}} \left( y \right) \ind{y \in \set{A}_{x}} \mathrm{d}P_{Y | X = x} \left( y \right)  \mathrm{d}P_{X}  \left(x\right) \spnum \\
\label{EqNovember16at18h36in2024NiceE}
& = &\int   \frac{\mathrm{d}P_{Y}}{\mathrm{d}P_{Y | X = x}} \left( y \right) \ind{y \in \set{A}_{x}} \mathrm{d}P_{YX}  \left(y, x\right) \spnum \\
\label{EqNovember16at18h36in2024NiceF}
& = &\int_{\hat{\set{A}}}   \frac{\mathrm{d}P_{Y}}{\mathrm{d}P_{Y | X = x}} \left( y \right) \mathrm{d}P_{YX}  \left(y, x\right) \spnum \\
\label{EqNovember16at18h36in2024NiceG}
& = &\int_{\set{A}}   \frac{\mathrm{d}P_{Y}}{\mathrm{d}P_{Y | X = x}} \left( y \right) \mathrm{d}P_{XY}  \left(x,y\right), \spnum 
\end{eqnarray}
where the sets~$\set{A}_{x}$,~$\set{A}_{y}$, and~$\hat{\set{A}}$ are defined in~\eqref{EqNovember15at16h57in2024InTheBusToNice},~\eqref{EqNovember15at16h39in2024InTheBusToNice}, and~\eqref{EqNovember16at18h49in2024Nice}; and the measure~$P_{YX}$ is defined in~\eqref{EqOctober30at7h48in2024SophiaAntipolis}.
%
The equality in~\eqref{EqNovember16at18h36in2024NiceC} follows by exchanging the order of the integrals \cite[Theorem~$2.6.6$]{ash2000probability}; 
the equality in~\eqref{EqNovember16at18h36in2024NiceC1} follows from Assumption~$(a)$ and \cite[Theorem~$2.2.3$]{lehmann2005testing}.

The proof is completed by noticing that from~\eqref{EqNovember16at18h29in2024NiceF} and~\eqref{EqNovember16at18h36in2024NiceG}, the following equalities hold:
%
\begin{eqnarray}
P_{X, Y }\left( \set{A} \right) \middlesqueezeequ & = &\int_{\set{A}} \frac{\mathrm{d}P_{X}}{\mathrm{d}P_{X | Y = y}} \left( x \right)    \mathrm{d}P_{XY} \left( x, y\right),  \\ 
& = & \int_{\set{A}}   \frac{\mathrm{d}P_{Y}}{\mathrm{d}P_{Y | X = x}} \left( y \right) \mathrm{d}P_{XY}  \left(x,y\right),
\end{eqnarray}
which together with \cite[Theorem~$2.2.3$]{lehmann2005testing} implies the equality in~\eqref{EqMay20at16h50in2024} almost surely with respect to the measure~$P_{XY} \in \triangle\left(\set{X} \times \set{Y} \right)$ in~\eqref{EqMay20at14h23in2024}. This completes the proof.

\end{IEEEproof}

\begin{remark}
An alternative proof for Theorem \ref {ThUnitMeasure} can be written as follows, by combining Theorems~\ref{TheoBYR}~\ref{TheoBYR2} and \ref{TheoInverseRND}.
\begin{eqnarray}
&& \int  \frac{\mathrm{d} P_{Y|X = x}}{\mathrm{d}P_{Y}} (y) \mathrm{d}P_{X}(x)\\
 &=& \int  \frac{\mathrm{d} P_{Y|X = x}}{\mathrm{d}P_{Y}} (y)\frac{\mathrm{d} P_{X}}{\mathrm{d}P_{X | Y= y}} (x) \mathrm{d}P_{X|Y = y}(x)\\\squeezeequ\spnum
 &=&  \int  \mathrm{d}P_{X|Y = y}(x) = 1
 \end{eqnarray}

\end{remark}

\begin{example}
%This exemples shows that in (a) the asumption on all x is necessary. Indeed, lets prove that there exist some x in omega such that PY given X is 0 and Py is not 0 in this case the absolute continuity assumption is not true therefore the random Nikiodym derivation of PY|x py does not exists. 

%From definition \ref{DefAbsCountinuity}, 
This example shows that there exist a $z \in \set{X}$ such that $P_{Y|X=z}(A) = 0$ and $P_Y(A) > 0$.

For all $x \in \mathbb{R}$, for $\set{A} \in \mathscr{F}_Y$, for $\set{B} \in \mathscr{F}_Y$ and $\mu$ the Lebesgue measure, $P_{Y|X=x}(A)$ and
$P_X (B)$ are defined as follows 
\begin{IEEEeqnarray}{rCl}
P_{Y|X=x}(A) &=& \ind{A \cap \{(x,y) \in \mathbb{R}^2: x=y \} \neq \emptyset}\\
P_X (B) &=& \mu(B)
\end{IEEEeqnarray}
%and 
%\begin{IEEEeqnarray}{rCl}
%P_X (B) &=& \mu(B)
%\end{IEEEeqnarray}

Given $A \in \set{B}(\mathbb{R})$ such that $P_Y(A) = 0$, then 
\begin{IEEEeqnarray}{rCl}
P_Y(A) &= &\int P_{Y|X=x}(A) \mathrm{d}P_X(x)\\
\label{EqLaCadiere9am59}
& = & \int \ind{A \cup \{(x,y) \in \mathbb{R}^2: x=y  \} \neq \emptyset} \mathrm{d}P_X(x)\\ 
\label{EqLaCadiere10am00}
& = & \int \ind{A \cup \{(x,x)  \} \neq \emptyset} \mathrm{d}P_X(x)\\ 
\label{EqLaCadiere10am01}
& = & \int \ind{x \in A} \mathrm{d}P_X(x)\\
\label{EqLaCadiere10am02}
& = &  \int_A \mathrm{d}P_X(x) \\
\label{EqLaCadiere10am03}
&=& \mu(A)
\end{IEEEeqnarray}

Let us define the infimum as follows, 
\begin{IEEEeqnarray}{rCl}
\label{Eqdefinf}
\underline{x} & = & inf_y \{x \in \mathbb{R}: (x,y) \in A\}
\end{IEEEeqnarray}
Then for all $z \in \mathbb{R}$ such that $z = \underline{x}-\epsilon$ for $\epsilon >0$, 


The equality in \eqref{EqMadrid9am59} follows from \eqref{eqmarginal1}; the equality \eqref{EqMadrid10am03} follows from the definition of the Lebesgue measure of a singleton. Thus, this exemple constructs $P_{Y|X}$ and $P_X$ such that for some $\set{A} \in \set{B}(\mathbb{R})$  but $P_{Y|X=x}(A) = 0$ but $P_Y(A) > 0$  for some $x \in \set{X}$ , therefore the Radom-Nikodym derivative of $P_Y$ with respect to $P_Y|X=x$ does not exist. This proves that assumption (a) in \ref{TheoBYR2} cannot be weakened and is needed for all $x \in \set{X}$. A example for assumption (b) in \ref{TheoBYR2} can be constructed similarly. 
\end{example}


\begin{example}
Let $\left( \mathbb{R}, \set{B}(\mathbb{R}), P_X \right)$ be a probability space, with $P_X$ absolutely continuous with respect to 
the Lebesgue measure $\mu$ on $\left(\mathbb{R}, \set{B}(\mathbb{R}) \right)$. For all $x \in \mathbb{R}$ and for all $\set{A} \in 
\set{B}(\mathbb{R})$, let 
\begin{IEEEeqnarray}{rCl}
P_{Y|X} &\triangleq& \{P_{Y|X=x} \in \triangle \left(\mathbb{R}, \set{B}(\mathbb{R}) \right) : x \in \mathbb{R} \}\\
\nonumber
\text{such that:}\\
&&P_{Y|X=x} (\set{A}) = \ind{x\in \set{A}}
\end{IEEEeqnarray}
Hence, there exists a unique measure on $\left(\mathbb{R}^2, \set{B}(\mathbb{R}) \right)$ denoted by $P_{XY}$ such that:

\begin{IEEEeqnarray}{rCl}
P_{XY}(\set{A}) &=& \int_{\mathbb{R}} \int_{\set{A}_y} \mathrm{d} P_{Y|X=x}(y) \mathrm{d} P_X(x)\\
&=& \int_{\mathbb{R}} P_{Y|X=x}(\set{A}_x) \mathrm{d} P_X(x)\\
&=& \int _{\mathbb{R}} \ind{(x,x) \in \set(A)} \mathrm{d} P_X(x)\\
&=& P_X \left( \{ t\in \mathbb{R}: (t,t) \in \set{A} \} \right).
\end{IEEEeqnarray}

Assuming that $\set{A} = \{ (x,y) \in \mathbb{R}^2: x=y\}$, it follows that 
\begin{IEEEeqnarray}{rCl}
P_{XY} &=& P_X \left( \{ t\in \mathbb{R}: (t,t) \in \set{A} \} \right)\\
&=& P_X(\mathbb{R})\\
&=& 1.
\end{IEEEeqnarray}
Moreover, let  the measure $P_Y$ on $\left( \mathbb{R}, \set{B}(\mathbb{R})\right)$ be such that for all $\set{A} \in \set{B}
(\mathbb{R})$, 
\begin{IEEEeqnarray}{rCl}
P_Y(\set{A}) &=& P_{XY} (\mathbb{R} \times A)\\
&=& P_X(\{t \in \mathbb{R}: (t,t) \in \set{A}\}) = P_X(\set{A}), 
\end{IEEEeqnarray}
hence $P_Y$ is absolutely continuous with respect to $\mu$. 
Hence,
\begin{IEEEeqnarray}{rCl}
&&P_XP_Y \left(\{(x,y) \in \mathbb{R}: x=y\} \right) \\
&=& \int_{\{(x,y)\in \mathbb{R}^2: x=y\}} \mathrm{d} P_XP_Y (x,y)\\
&=&  \int_{\{(x,y)\in \mathbb{R}^2: x=y\}} \frac{\mathrm{d} P_XP_Y}{\mathrm{d} \mu \mu}(x,y) \mathrm{d} \mu \mu (x,y)\\
&=&  \int_{\{(x,y)\in \mathbb{R}^2: x=y\}} \frac{\mathrm{d} P_X}{\mathrm{d} \mu}(x)\frac{\mathrm{d} P_Y}{\mathrm{d} \mu}(y) \mathrm{d} \mu \mu (x,y)\\
&=& \int_{\mathbb{R}} \int_{\mathbb{R}} \ind{\{ x=y\}} \frac{\mathrm{d} P_X}{\mathrm{d} \mu}(x)\frac{\mathrm{d} P_Y}{\mathrm{d} \mu}(y) \mathrm{d} \mu (x) \mathrm{d} \mu (y)\\
\label{AntibesJan23pm2025}
&=& \int_{\mathbb{R}} \int_{\{y\}} \frac{\mathrm{d} P_X}{\mathrm{d} \mu}(x)\frac{\mathrm{d} P_Y}{\mathrm{d} \mu}(y) \mathrm{d} \mu (x) \mathrm{d} \mu (y)\\
&=& 0
\end{IEEEeqnarray}

where the equality in \eqref{AntibesJan23pm2025} follows from the fact that for all $y \in \mathbb{R}, \mu(\{y\}) = 0$. Thus 
$P_{XY}$ is not absolutely continuous with respect to $P_XP_Y$.
\end{example}

%%%%%%%%%%%%%%%%%%%%%
%. Information measures
%%%%%%%%%%%%%%%%%%%%%
\section{Information measures}\label{DefInformation} 
\begin{definition} [Information]
Given a random variable $X$ following a probability measure $P_X \in \triangle(\set{X}, \set{F}_\set{X})$ and a sigma finite probability measure $Q \in \triangle(\set{X}, \set{F}_\set{X})$, with $P_X\ll Q$, for all $x \in suppP_X$, the information provided by $x$ is
\begin{eqnarray}
\iota_X(x) & =& - \log \RND{P_X}{Q}(x).
\end{eqnarray}
When $Q$ is the counting measure $\RND{P_X}{Q}$ is the probability mass function induced by $P_X$, and when $Q$ is the Lebesgue measure $\RND{P_X}{Q}$ is the probability density function induced by $P_X$.
\end{definition}

\begin{definition}[Information Spectrum]\label{DefInfoSpectrum}  
Given a random variable $X$ following a probability measure $P_X \in \triangle(\set{X}, \set{F}_\set{X})$ and the counting measure $Q \in \triangle(\set{X}, \set{F}_\set{X}) $, with $P_X\ll Q$, the information spectrum of $X$, denoted by $S_X: \mathds{R} \rightarrow [0,1]$, is the cumulative distribution function of the random variable $\iota_{X}(X)$. That is, for all $a \in \mathds{R}$,
\begin{equation}
S_X(a) = \sum_{x \in \supp{P_X}} \RND{P_X}{Q}(x) \ind{\iota_{X}(x) \leqslant a}.
\end{equation}
\end{definition}

\begin{definition}[Joint Information]\label{DefJointInformation} 
Given two random variables $X$ and $Y$  with joint probability mass function $p_{X Y}: \mathds{R}^2 \rightarrow [0,1]$, for all $(x,y) \in \supp{p_X} \times \supp{p_Y}$, the information provided by  $ (x,y)$ is  
\begin{equation}
\label{EqDefInfo}
\iota_{X Y }(x,y) = -\log\left( p_{X Y }(x, y) \right) .
\end{equation}
\end{definition}

\subsection{KL}


\subsection{Mutual information}

Let $P_{XY} \in \triangle\left( \set{X} \times \set{Y} \right)$ and $P_{YX} \in 
\triangle\left( \set{Y} \times \set{X} \right)$ be two joint probability measures, satisfying \eqref{EqJanuary10at16h43in2025}, such 
that $P_{XY} \triangleq P_{X|Y}P_X$ and $P_{XY} \triangleq P_{Y|X}P_Y$. Let the marginal probability measures in $
\triangle\left( \set{X} \right)$ and $\triangle\left( \set{Y} \right)$, be denoted by $P_X$ and $P_Y$ and satisfy for all measurable 
sets $\set{A} \subseteq \set{X}$ and for all measurable sets $\set{B} \subseteq \set{Y}$, \eqref{eqmargpx} and \eqref{eqmargpy} 
respectively.. Let two product measures $P_XP_Y \in \triangle\left( \set{Y} \times \set{X} \right) $and $P_YP_X \in 
\triangle\left( \set{Y} \times \set{X} \right)$ be the product of the 
marginals. For all~$x \in \set{X}$, the probability measure~$P_{Y | X = x}$ is absolutely continuous with respect to~$P_{Y}$; 
and for all~$y \in \set{Y}$, the probability measure~$P_{X | Y = y}$ is absolutely continuous with respect to~$P_{X}$.

\begin{IEEEeqnarray}{rCl}
\mathrm{I}(P_{Y|X}; P_Y P_X) &\triangleq& \mathrm{D}\left(P_{YX} \| P_Y P_X\right) \\
&=& \int \log \left( \frac{\mathrm{d}P_{YX}}{\mathrm{d}P_Y P_X}(x, y) \right)  \mathrm{d}P_{YX}(x, y) \\
&=& \int \mathrm{D}\left(P_{Y|X=x} \| P_Y\right)  \mathrm{d}P_X(x),
\end{IEEEeqnarray}
\begin{IEEEeqnarray}{rCl}
\mathrm{I}(P_{X|Y}; P_X P_Y) &\triangleq& \mathrm{D}\left(P_{XY} \| P_X P_Y\right) \\
&=& \int \log \left( \frac{\mathrm{d}P_{XY}}{\mathrm{d}P_X P_Y}(x, y) \right)  \mathrm{d}P_{XY}(x, y) \\
&=& \int \mathrm{D}\left(P_{X|Y=y} \| P_X\right)  \mathrm{d}P_Y(y).
\end{IEEEeqnarray}

\begin{IEEEeqnarray}{rCl}
&& \mathrm{D}\left(P_{XY} \| P_X P_Y\right) \\
&=& \int \log \left( \frac{\mathrm{d}P_{XY}}{\mathrm{d}P_X P_Y}(x, y) \right) \mathrm{d}P_{XY}(x, y) \\
&=& \int \log \left( \frac{\mathrm{d}P_{X|Y=y}}{\mathrm{d}P_X}(x) \right) \mathrm{d}P_{XY}(x, y) \\
&=& \int \log \left( \frac{\mathrm{d}P_{X|Y=y}}{\mathrm{d}P_X}(x)\right)   \frac{\mathrm{d}P_{XY}}{\mathrm{d}P_X P_Y}(x, y)  
\mathrm{d}P_X P_Y(x, y) \squeezeequ\spnum \\
&=& \int \log \left( \frac{\mathrm{d}P_{X|Y=y}}{\mathrm{d}P_X}(x) \right)  \frac{\mathrm{d}P_{X|Y=y}}{\mathrm{d}P_X}(x) 
\mathrm{d}P_XP_Y(x,y) \squeezeequ\spnum \\
%&=& \int \log \left( \frac{\mathrm{d}P_{X|Y=y}}{\mathrm{d}P_X}(x) \right)  \frac{\mathrm{d}P_{X|Y=y}}{\mathrm{d}P_X}(x)  
%\mathrm{d}P_X(x)  \mathrm{d}P_Y(y) \\\squeezeequ\spnum
&=& \int \int \log \left( \frac{\mathrm{d}P_{X|Y=y}}{\mathrm{d}P_X}(x) \right)  \frac{\mathrm{d}P_{X|Y=y}}{\mathrm{d}P_X}(x)  
\mathrm{d}P_X(x)  \mathrm{d}P_Y(y).\squeezeequ\spnum
\end{IEEEeqnarray}

Option 1:
\begin{IEEEeqnarray}{rCl}
\nonumber
&& \int \int \log \left( \frac{\mathrm{d}P_{X|Y=y}}{\mathrm{d}P_X}(x) \right)  \frac{\mathrm{d}P_{X|Y=y}}{\mathrm{d}P_X}(x)  
\mathrm{d}P_X(x)  \mathrm{d}P_Y(y)\squeezeequ\spnum \\
&=& \int \int \log \left( \frac{\mathrm{d}P_{X|Y=y}}{\mathrm{d}P_X}(x) \right)  \mathrm{d}P_{X|Y=y}(x)  \mathrm{d}P_Y(y) 
\squeezeequ\spnum \\
&=& \int \mathrm{D} \left( P_{X|Y=y} \| P_X \right)  \mathrm{d}P_Y(y).\squeezeequ\spnum
\end{IEEEeqnarray}

Option 2:

\begin{IEEEeqnarray}{rCl}
\nonumber
&& \int \int \log \left( \frac{\mathrm{d}P_{X|Y=y}}{\mathrm{d}P_X}(x) \right)  \frac{\mathrm{d}P_{X|Y=y}}{\mathrm{d}P_X}(x)  
\mathrm{d}P_X(x)  \mathrm{d}P_Y(y)\squeezeequ\spnum \\
%&=& \int \int \log \left( \frac{\mathrm{d}P_{X|Y=y}}{\mathrm{d}P_X}(x) \right) \frac{\mathrm{d}P_{Y|X=x}}{\mathrm{d}P_Y}(y)  
%\mathrm{d}P_X(x)  \mathrm{d}P_Y(y) \\\squeezeequ\spnum
&=& \int \int \log \left(\frac{\mathrm{d}P_{Y|X=x}}{\mathrm{d}P_Y}(y) \right) \frac{\mathrm{d}P_{Y|X=x}}{\mathrm{d}P_Y}(y)  
\mathrm{d}P_X(x)  \mathrm{d}P_Y(y) \squeezeequ\spnum \\
&=& \int \int \log \left(\frac{\mathrm{d}P_{Y|X=x}}{\mathrm{d}P_Y}(y) \right) \frac{\mathrm{d}P_{Y|X=x}}{\mathrm{d}P_Y}(y)  
\mathrm{d}P_Y(y)  \mathrm{d}P_X(x) \squeezeequ\spnum \\
&=& \int \int \log \left( \frac{\mathrm{d}P_{Y|X=x}}{\mathrm{d}P_Y}(y)  \right)  \mathrm{d}P_{Y|X=x}(y)  \mathrm{d}P_X(x) 
\squeezeequ\spnum \\
%&=& \int \int \log \left( \frac{\mathrm{d}P_{Y|X=x}}{\mathrm{d}P_Y}(y) \right)  \mathrm{d}P_{Y|X=x}(y)  \mathrm{d}P_X(x) \\
%\squeezeequ\spnum
&=& \int \mathrm{D} \left( P_{Y|X=x} \| P_Y \right)  \mathrm{d}P_X(x).\squeezeequ\spnum
\end{IEEEeqnarray}

Option 3:
\begin{IEEEeqnarray}{rCl}
\nonumber
&& \int \int \log \left( \frac{\mathrm{d}P_{X|Y=y}}{\mathrm{d}P_X}(x) \right) \frac{\mathrm{d}P_{X|Y=y}}{\mathrm{d}P_X}(x)  
\mathrm{d}P_X(x)  \mathrm{d}P_Y(y) \squeezeequ\spnum \\
%&=& \int \int \log \left( \frac{\mathrm{d}P_{YX}}{\mathrm{d}P_Y P_X}(y, x) \right) \frac{\mathrm{d}P_{X|Y=y}}{\mathrm{d}P_X}
%(x)  \mathrm{d}P_X(x)  \mathrm{d}P_Y(y) \\\squeezeequ\spnum
&=& \int \int \log \left( \frac{\mathrm{d}P_{YX}}{\mathrm{d}P_Y P_X}(y, x) \right) \frac{\mathrm{d}P_{YX}}{\mathrm{d}P_Y P_X}(y, 
x)  \mathrm{d}P_X(x)  \mathrm{d}P_Y(y)\squeezeequ\spnum \\
&=& \int \int \log \left( \frac{\mathrm{d}P_{YX}}{\mathrm{d}P_Y P_X}(y, x) \right) \frac{\mathrm{d}P_{YX}}{\mathrm{d}P_Y P_X}(y, 
x)  \mathrm{d}P_Y(y)  \mathrm{d}P_X(x) \squeezeequ\spnum \\
&=& \int \log \left( \frac{\mathrm{d}P_{YX}}{\mathrm{d}P_Y P_X}(y, x) \right) \frac{\mathrm{d}P_{YX}}{\mathrm{d}P_Y P_X}(y, x)  
\mathrm{d}P_{Y}P_{X}(y, x) \squeezeequ\spnum \\
&=& \int \log \left( \frac{\mathrm{d}P_{YX}}{\mathrm{d}P_Y P_X}(y, x) \right)  \mathrm{d}P_{YX}(y, x) \\
&=& \mathrm{D}\left( P_{YX} \| P_Y P_X \right).\squeezeequ\spnum
\end{IEEEeqnarray}

\subsection{Lautum information}


%\triggeratref{24}
\bibliographystyle{IEEEtranlink}%tran
%\bibliographystyle{../../WeeklyReport/Bibliography/Mytran}
\bibliography{Bermudez-reference.bib}


%\section{Biography Section}
%%
%\begin{biographynophoto}{Francisco Daunas} received the M.Sc. degree in Advanced Control and Systems Engineering from The University of Sheffield, United Kingdom, in 2019. He is currently a Ph.D. candidate at the Department of Automatic Control and Systems Engineering of The University of Sheffield, United Kingdom. His research interests span information theory, statistical learning, and machine learning.
%\end{biographynophoto}
%\balance
%
%\begin{biographynophoto}{Iñaki Esnaola} (Member, ) received the M.Sc. degree in Electrical Engineering from University of Navarra, Spain in 2006 and a Ph.D. in Electrical Engineering from University of Delaware, Newark, DE in 2011. He is currently a Senior Lecturer in the Department of Automatic Control and Systems Engineering of The University of Sheffield, and a Visiting Research Collaborator in the Department of Electrical Engineering of Princeton University, Princeton, NJ. In 2010-2011 he was a Research Intern with Bell Laboratories, Alcatel-Lucent, Holmdel, NJ, and in 2011-2013, he was a Postdoctoral Research Associate at Princeton University, Princeton, NJ. His research interests include information theory and communication theory with an emphasis on the application to smart grid problems.
%\end{biographynophoto}
%
%
%\begin{biographynophoto}{Samir M. Perlaza} (Senior Member, ) is a research scientist with the Institut National de Recherche en Informatique et en Automatique (INRIA), France, and a visiting research scholar at the Department of Electrical Engineering of Princeton University, Princeton, (NJ, USA). He received the M.Sc. and Ph.D. degrees from \'Ecole Nationale Sup\'erieure des T\'el\'ecommunications (Telecom ParisTech), Paris, France, in 2008 and 2011, respectively. Previously, from 2008 to 2011, he was a Research Engineer at France T\'el\'ecom - Orange Labs (Paris, France). He has held long-term academic appointments at the Alcatel-Lucent Chair in Flexible Radio at Supélec (Gif-sur-Yvette, France); at Princeton University (Princeton, NJ) and at the University of Houston (Houston, TX). Dr. Perlaza currently serves as an Editor of the  Transactions on Communications. Dr. Perlaza has been distinguished by the European Commission with an Alban Fellowship in 2006 and a Marie Skłodowska-Curie Fellowship in 2015. His research interests lie in the overlap of signal processing, information theory, game theory and wireless communications.
%\end{biographynophoto}
%
%\begin{biographynophoto}{H. Vincent Poor}(Life Fellow, ) received the Ph.D. degree in electrical engineering and computer science from Princeton University in 1977. From 1977 until 1990, he was on the faculty of the University of Illinois at Urbana-Champaign. Since 1990, he has been on the faculty at Princeton, where he is currently the Michael Henry Strater University Professor of Electrical Engineering. During 2006 to 2016, he served as the dean of Princeton’s School of Engineering and Applied Science. He has also held visiting appointments at several other universities, including most recently at Berkeley and Cambridge. His research interests include the areas of information theory, machine learning and network science, and their applications in wireless networks, energy systems and related fields. Among his publications in these areas is the forthcoming book Machine Learning and Wireless Communications (Cambridge University Press, 2021). He is a member of the National Academy of Engineering and the National Academy of Sciences, and is a foreign member of the Chinese Academy of Sciences, the Royal Society, and other national and international academies. Recent recognition of his work includes the 2017  Alexander Graham Bell Medal and a DEng honoris causa from the University of Waterloo, awarded in 2019.
%\end{biographynophoto}



\end{document}


